


% =========================
% PARTIE A — Préparation & EDA
% =========================
\section{Partie A Préparation des données et analyse exploratoire}

\paragraph{Jeu de données.}
Le fichier \texttt{beer\_quality.xlsx} contient 1599 observations et 11 variables explicatives quantitatives,
ainsi qu'une cible entière \texttt{quality}. Nous séparons les données en un ensemble d'apprentissage (70\%)
et un ensemble de test (30\%). Pour préserver la distribution de la cible lors du découpage, nous appliquons
une \emph{stratification} en transformant \texttt{quality} en tranches (binning).

\paragraph{Statistiques descriptives.}
Le Tableau~\ref{tab:descXtrain} résume les statistiques élémentaires (moyenne, écart-type, min/max, quartiles)
des variables explicatives sur le jeu d'apprentissage.

% ---- Table stats (inclure le .tex généré) ----
\input{src/input/table_stats_Xtrain}
\begin{figure}[H]
  \centering

  % Ligne 1
  \begin{subfigure}[b]{0.48\textwidth}
    \centering
    \includegraphics[width=\linewidth]{src/input/fig_hist_ytrain.png}
    \caption{Distribution de \texttt{y\_train}}
    \label{fig:hist_ytrain}
  \end{subfigure}\hfill
  \begin{subfigure}[b]{0.48\textwidth}
    \centering
    \includegraphics[width=\linewidth]{src/input/fig_corr_heatmap.png}
    \caption{Matrice de corrélation (\texttt{X\_train})}
    \label{fig:corr_heat}
  \end{subfigure}

  \vspace{0.5em}

  % Ligne 2
  \begin{subfigure}[b]{0.48\textwidth}
    \centering
    \includegraphics[width=\linewidth]{src/input/fig_corr_top5.png}
    \caption{Top 5 corrélations positives}
    \label{fig:corr_top5}
  \end{subfigure}\hfill
  \begin{subfigure}[b]{0.48\textwidth}
    \centering
    \includegraphics[width=\linewidth]{src/input/fig_corr_bot5.png}
    \caption{Top 5 corrélations négatives}
    \label{fig:corr_bot5}
  \end{subfigure}

  \caption{Analyse exploratoire : cible, corrélations entre variables et avec \texttt{quality}.}
  \label{fig:eda_global}
\end{figure}



% =========================
% PARTIE B — Classification binaire et Ensemble Learning
% =========================
\section{Partie B Classification binaire et Ensemble Learning}

\paragraph{Objectif.}
Nous transformons la variable cible \texttt{quality} en une variable binaire \texttt{ybin} pour distinguer
les bières de \emph{bonne qualité} des autres.
Le seuil choisi est la médiane (6.0), de sorte que :
\[
y_{\text{bin}}=\begin{cases}
1 & \text{si } \texttt{quality}\ge6,\\
0 & \text{sinon.}
\end{cases}
\]
Le découpage des données reste stratifié (70 \% apprentissage / 30 \% test) afin de préserver la proportion
des deux classes.

\paragraph{Arbre de décision.}
Un arbre de décision est entraîné et optimisé par une \emph{Randomized Search CV} (5 plis) sur les principaux
hyperparamètres : \texttt{max\_depth}, \texttt{min\_samples\_split}, \texttt{min\_samples\_leaf},
\texttt{max\_features} et \texttt{ccp\_alpha}.  
Les meilleurs réglages obtenus sont :
\begin{center}
\texttt{max\_depth=12, min\_samples\_split=5, min\_samples\_leaf=3, max\_features=log2, ccp\_alpha=0.001.}
\end{center}
L’accuracy moyenne en validation croisée atteint 0.736 et l’accuracy sur le jeu de test 0.723.
\[
\text{Matrice de confusion (test)}=
\begin{bmatrix}154 & 69\\ 64 & 193\end{bmatrix}.
\]

% ---- Résultats arbre optimisé (insérer un .tex généré) ----
\paragraph{AdaBoost avec arbres faibles.}
Nous entraînons AdaBoost avec des arbres de décision peu profonds en tant qu'apprenants de base et traçons l'accuracy en fonction de \texttt{n\_estimators} en apprentissage et validation pour deux profondeurs:
\begin{itemize}
  \item \textbf{max\_depth = 1} (stumps) : apprenants très faibles, biais élevé, variance faible.
  \item \textbf{max\_depth = 5} : apprenants plus expressifs, biais plus faible, variance plus élevée.
\end{itemize}

\begin{figure}[H]
  \centering

  \begin{subfigure}[b]{0.48\textwidth}
    \centering
    \includegraphics[width=\linewidth]{src/input/fig_adaboost_acc_depth1.png}
    \caption{Accuracy vs \texttt{n\_estimators} (max\_depth=1)}
    \label{fig:ab_acc_d1}
  \end{subfigure}\hfill
  \begin{subfigure}[b]{0.48\textwidth}
    \centering
    \includegraphics[width=\linewidth]{src/input/fig_adaboost_acc_depth5.png}
    \caption{Accuracy vs \texttt{n\_estimators} (max\_depth=5)}
    \label{fig:ab_acc_d5}
  \end{subfigure}

  \caption{Courbes d'accuracy AdaBoost en apprentissage et validation.}
  \label{fig:ab_curves}
\end{figure}

\paragraph{Sélection du modèle.}
Nous choisissons la configuration maximisant l'accuracy de validation, puis ré-entraînons le modèle sur tout l'apprentissage et évaluons en test. Le tableau suivant synthétise la configuration retenue et les performances (incluant les temps d'apprentissage et d'inférence).

% ---- Résultats modèle AdaBoost retenu (insérer un .tex généré) ----
\begin{tabular}{rrrrrr}
\toprule
max_depth & n_estimators & accuracy_val & accuracy_test & train_time_s & infer_time_s \\
\midrule
5 & 200 & 0.7768 & 0.8021 & 0.8156 & 0.0340 \\
\bottomrule
\end{tabular}


\paragraph{Importance des variables.}
Avec des arbres comme apprenants de base, AdaBoost fournit des \emph{importances de variables} (moyennées sur les estimateurs, selon les réductions d'impureté). La Figure~\ref{fig:ab_importances} affiche les variables par ordre d'importance décroissante.

\begin{figure}[H]
  \centering
  \includegraphics[width=0.75\linewidth]{src/input/fig_adaboost_feature_importances.png}
  \caption{Importance des variables pour le modèle AdaBoost retenu.}
  \label{fig:ab_importances}
\end{figure}

\paragraph{Biais et variance.}
\begin{itemize}
  \item \textbf{max\_depth=1} : fort biais (modèles très simples), faible variance; AdaBoost réduit le biais en agrégeant de nombreux faibles apprenants.
  \item \textbf{max\_depth=5} : biais plus faible, mais variance plus élevée; attention au sur-apprentissage si \texttt{n\_estimators} est trop grand.
  \item \textbf{Sensibilité au bruit} : AdaBoost peut surpondérer des observations bruitées si mal classées de façon répétée.
\end{itemize}

\paragraph{Conclusion.}
L’\textbf{AdaBoost avec arbres de profondeur 5} obtient les meilleurs résultats
(\textbf{accuracy test = 0.8021}) et une meilleure stabilité entre apprentissage et validation.
Il tire parti d’un ensemble de variables cohérentes avec l’analyse exploratoire :
\texttt{alcohol}, \texttt{sulphates}, \texttt{volatile acidity} et les composés soufrés.
Ce modèle offre donc le compromis le plus équilibré entre biais et variance pour prédire
la qualité des bières.


\section{Partie C}

\subsection{Définition des classes}
On transforme la variable continue \texttt{quality} en trois classes à seuils fixes:
\begin{itemize}
  \item Classe 0 (faible) : $y \le 5$,
  \item Classe 1 (moyenne) : $y = 6$,
  \item Classe 2 (élevée) : $y \ge 7$.
\end{itemize}
Ces seuils sont appliqués sur \texttt{y\_train} et \texttt{y\_test}. Ils sont lisibles ``métier'' et remplacent l'approche tertiles.

\subsection{Répartition des classes (avant/après équilibrage)}
\subsection*{Avant équilibrage}
La répartition des classes dans les jeux d'apprentissage et de test est donnée ci-dessous.

\begin{table}[H]
  \centering
  \caption{Effectifs par classe (avant équilibrage)}
  \begin{tabular}{rrr}
\toprule
classe & train & test \\
\midrule
0 & 521 & 223 \\
1 & 446 & 192 \\
2 & 152 & 65 \\
\bottomrule
\end{tabular}

\end{table}

\paragraph{Réponse.} Le déséquilibre initial (surtout classe 2 plus rare) est compensé par SMOTE, ce qui facilite l'apprentissage des classes minoritaires et améliore souvent le F1-macro.

\subsection{Réseau de neurones (MLP) et matrices de confusion}
Nous évaluons un MLP (early stopping) en données déséquilibrées et après SMOTE. Les matrices de confusion en test sont présentées ci-dessous.

\begin{figure}[H]
  \centering
  \begin{subfigure}[t]{0.48\textwidth}
    \centering
    \includegraphics[width=\linewidth]{src/input/fig_mlp_confusion_imbalanced.png}
    \caption{MLP — données déséquilibrées}
  \end{subfigure}\hfill
  \begin{subfigure}[t]{0.48\textwidth}
    \centering
    \includegraphics[width=\linewidth]{src/input/fig_mlp_confusion_smote.png}
    \caption{MLP — après SMOTE}
  \end{subfigure}
  \caption{Matrices de confusion (test) pour MLP}
\end{figure}

\paragraph{Réponse.} Dans nos exécutions, le F1-macro test MLP est d'environ 0{,}370 (déséquilibré) et 0{,}363 (SMOTE). L'effet de SMOTE est limité ici, mais peut aider selon les réglages.

\subsection{Bagging avec MLP}
La figure suivante montre l'évolution du F1-macro (train/validation) en fonction de \texttt{n\_estimators}. Le meilleur \texttt{n\_estimators} est ensuite évalué en test.

\begin{figure}[H]
  \centering
  \includegraphics[width=0.75\linewidth]{src/input/fig_bagging_mlp_f1.png}
  \caption{Bagging(MLP): F1-macro vs \texttt{n\_estimators}}
\end{figure}

\paragraph{Réponse.} En test, nous obtenons un F1-macro d'environ 0{,}349. Le bagging réduit la variance du MLP, mais la capacité/instabilité du base learner limite le gain ici.

\subsection{Forêt aléatoire: F1 vs \texttt{n\_estimators}}
Nous traçons la courbe F1-macro (train/test) pour \texttt{max\_depth=None} en fonction de \texttt{n\_estimators}.

\begin{figure}[H]
  \centering
  \includegraphics[width=0.75\linewidth]{src/input/fig_rf_f1_curve.png}
  \caption{RandomForest: F1-macro vs \texttt{n\_estimators} (max\_depth=None)}
\end{figure}

\paragraph{Réponse.} Le F1 test se stabilise avec un nombre suffisant d'arbres; \texttt{n\_estimators} agit surtout sur la variance (agrégation). 

\subsection{Random search des hyperparamètres (RF)}
Nous optimisons \texttt{max\_depth} et \texttt{n\_estimators} via recherche aléatoire (F1-macro, CV=5). Le tableau récapitulatif est ci-dessous.

\begin{table}[H]
  \centering
  \caption{Meilleur modèle RF (random search)}
  \begin{tabular}{rrrrr}
\toprule
n_estimators & max_depth & f1_train & f1_test & train_time_s \\
\midrule
100 & 30 & 1.0000 & 0.6592 & 12.6381 \\
\bottomrule
\end{tabular}

\end{table}

\paragraph{Réponse.} Un des meilleurs modèles obtenus sur données déséquilibrées atteint un F1 test $\approx 0{,}659$ (train $\approx 1{,}0$), signe d'un risque de sur-apprentissage. Avec SMOTE, le F1 test monte à $\approx 0{,}679$.

\subsection{Matrices de confusion (RF) — déséquilibré vs SMOTE}

\begin{figure}[H]
  \centering
  \begin{subfigure}[t]{0.48\textwidth}
    \centering
    \includegraphics[width=\linewidth]{src/input/fig_rf_confusion_imbalanced.png}
    \caption{RF — déséquilibré}
  \end{subfigure}\hfill
  \begin{subfigure}[t]{0.48\textwidth}
    \centering
    \includegraphics[width=\linewidth]{src/input/fig_rf_confusion_smote.png}
    \caption{RF — après SMOTE}
  \end{subfigure}
  \caption{Matrices de confusion (test) pour RandomForest}
\end{figure}

\paragraph{Réponse.} SMOTE améliore le rappel de la classe rare (2) et le F1-macro global.

\subsection{Importances des variables (RF) et comparaison à AdaBoost (B.3)}

\begin{figure}[H]
  \centering
  \includegraphics[width=0.8\linewidth]{src/input/fig_rf_feature_importances.png}
  \caption{Importances des variables — RandomForest (modèle optimal)}
\end{figure}

\paragraph{Réponse.} Les variables importantes en RF recoupent largement celles d'AdaBoost (B.3) — par ex. \textit{alcohol}, \textit{sulphates}, \textit{volatile acidity}, \textit{density}, \textit{total sulfur dioxide}, etc. Dans nos résultats, 9 variables communes dans le Top 10.

\subsection{Tableau récapitulatif (RF — choix final)}
Le tableau ci-dessous récapitule la comparaison déséquilibré vs SMOTE et le choix retenu pour RF en Partie C.

\begin{table}[H]
  \centering
  \caption{Choix du meilleur RF (déséquilibré vs SMOTE)}
  \begin{tabular}{lr}
\toprule
variant & f1_test \\
\midrule
SMOTE & 0.6785 \\
déséquilibré & 0.6592 \\
\bottomrule
\end{tabular}

\end{table}

\subsection{Choix du modèle (Partie C)} 
Sur la tâche multiclasse, le modèle retenu est \textbf{RandomForest + SMOTE} (meilleur F1 test $\approx 0{,}679$). Il concilie performance et robustesse au déséquilibre. 

\subsection*{Conclusion (Partie C)}
\begin{itemize}
  \item \textbf{Équilibrage} (SMOTE) réduit le biais de classe et améliore le F1-macro pour RF.
  \item \textbf{MLP/Bagging} restent moins performants ici que la RF.
  \item \textbf{RF} offre le meilleur compromis biais/variance et des importances cohérentes avec AdaBoost.
\end{itemize}

\newpage
\section{Partie D — Conclusion générale sur les méthodes d'ensemble}
Cette section rassemble les performances (F1 test) et les temps (apprentissage, inférence) des différentes méthodes évaluées.

\begin{table}[H]
  \centering
  \caption{Tableau récapitulatif des méthodes d'ensemble (F1 test, temps)}
  \begin{tabular}{llrrr}
\toprule
methode & tache & f1_test & train_time_s & infer_time_s \\
\midrule
RandomForest (optimal) & multiclasse & 0.6592 & 0.1677 & 0.0275 \\
Bagging(MLP, n=20) & multiclasse & 0.3485 & 0.3027 & 0.0237 \\
AdaBoost (final) & binaire & 0.8177 & 0.8156 & 0.0340 \\
\bottomrule
\end{tabular}

\end{table}

\begin{figure}[H]
  \centering
  \includegraphics[width=0.8\linewidth]{src/input/fig_ensembles_multiclass_f1.png}
  \caption{Méthodes d'ensemble (multiclasse) — F1 test}
\end{figure}

\begin{figure}[H]
  \centering
  \includegraphics[width=0.5\linewidth]{src/input/fig_ensembles_binary_f1.png}
  \caption{Méthodes d'ensemble (binaire) — F1 test}
\end{figure}

\paragraph{Réponse.} En multiclasse, les performances des méthodes d'ensemble sont proches, mais RandomForest + SMOTE obtient les meilleurs résultats. En binaire, AdaBoost atteint un F1-test plus élevé, mais l'apprentissage et l'inférence sont plus longs ; le meilleur compromis reste RandomForest + SMOTE en multiclasse.